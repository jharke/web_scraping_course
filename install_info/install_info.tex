\documentclass[a4paper]{article}
\usepackage[a4paper, top=1in, bottom=1in, left=1in, right=1in]{geometry}
\setlength{\parskip}{0.8em}
\setlength\parindent{0pt}
\usepackage{hyperref}
\usepackage{graphicx}
\usepackage{enumitem}

\usepackage{xcolor}

\usepackage{minted}
\definecolor{solarized-light-bg}{HTML}{FDF6E3}
\setminted{
  style=solarized-light,
  bgcolor=solarized-light-bg,
  python3=true
}
\usepackage{bookmark}

\begin{document}

\includegraphics[width=80pt]{../misc/bccp_logo_transparent.png}
\vspace{1cm}

\begin{center}
\textbf{\Large BCCP Web Scraping Course}

{\large Preparations for the Course}
\vspace{0.3cm}

Julian Harke and Kevin Tran\footnote{Julian is a Research Fellow at the WZB and a PhD student at Vrije Universiteit Amsterdam. Kevin is a PhD student at the DIW Graduate Center and Technische Universit\"at Berlin. Both are members of the Berlin Centre for Consumer Policies (BCCP).} \vspace{0.3cm}

June 24, 2019, 09:30 - 12:30 \\
June 25, 2019, 09:30 - 12:30 \\
June 26, 2019, 14:00 - 17:00
\end{center}

We would like to make the course as interactive as possible. So if you would like, please bring your computers and try out the prepared scripts on your own. In order to minimize troubleshooting time during the course, here are some things that could prepare before the first session.

\section{Install a Python 3 distribution and the needed packages}

As we are using Python during the course, you will need a working installation of Python (ideally Python 3.7). Installing Python can cause various complications on bad days, so doing this before the course is a good idea.

Two good Python distributions are Anaconda (\url{https://www.anaconda.com/distribution/}) and Miniconda (\url{https://docs.conda.io/en/latest/miniconda.html}). Anaconda has the advantage that it has many pre-installed packages. Miniconda has the advantage that it only comes with the most necessary packages and therefore requires less disk space. You can then easily install the packages you need manually. If you decide on one of these distributions, you can follow these steps:

\paragraph{1. Download the installer} Download the Python 3.7 installer for your operation system for Anaconda (\url{https://www.anaconda.com/distribution/}) or Miniconda (\url{https://docs.conda.io/en/latest/miniconda.html}).

\paragraph{2. Run the installer} Run the downloaded file and follow the instructions.

\paragraph{3. Create a Python environment} This step is not absolutely necessary but we do recommend that you create a Python environment for the course. A Python environment is like a virtual separate Python installation with only the packages you specify. You could also just install all needed packages in your root environment but this may lead to some conflicts sometimes.
\pagebreak{}

In order to create a Python environment proceed as follows:
\begin{enumerate}
	\item Open the Terminal/Command Prompt. Alternatively, you can also use the Anaconda Prompt.
	\item Create a Python 3.7 environment and install all needed packages on the go by typing and entering:
\begin{minted}[breaklines, breaksymbol=]{bat}
conda create -n py37 python=3.7 requests beautifulsoup4 selenium pandas lxml jupyter tweepy world_bank_data pandas-datareader
\end{minted}
        This will create an environment called ``py37'' (you can call it however you like) with a Python 3.7 install and the packages needed for the course.
	\begin{itemize}
		\item If you already have a Python environment or you would just like to root your root environment, you can also simply install the packages by typing and entering:
\begin{minted}[breaklines, breaksymbol=]{bat}
conda install requests beautifulsoup4 selenium pandas lxml jupyter tweepy world_bank_data pandas-datareader
\end{minted}
	\end{itemize}
	\item Test if the environment works by activating it using \mintinline{bat}{conda activate py37}
\end{enumerate}

\section{Installing a browser and downloading the corresponding driver}

For the part on browser automation, you will need to have an internet browser installed and downloaded a corresponding browser driver. In principle, you can use any browser you want given that a driver exists for it.

In the example script, I will use Google's Chrome browser. Doing the same will reduce the likelihood of browser-specific errors. For the course, please prepare the following:

\paragraph{1. Install and locate a browser} Decide on a browser and install it, if you do not already have it on your computer. I will use Google Chrome. Further, find out where the program file of your browser is located on your computer. You will need this to run the script on browser automation.

\paragraph{2. Download a fitting browser driver} Download a driver that fits your browser. Links to the drivers for Chrome, Edge, Firefox, and Safari can be found here: \url{https://selenium-python.readthedocs.io/installation.html#drivers}. Also make sure that the driver version fits your installed browser version. The easiest way to do so is to install/update to the most current version of the browser and download the most current version of the driver. Save the driver somewhere where you will find it later, you will need the location of the file to run the script on browser automation.

\section{Feel free to ask}

If you try any of these steps and get an error, feel free to contact any of us (\href{mailto:julian.harke@wzb.eu}{julian.harke@wzb.eu} or \href{mailto:ktran@diw.de}{ktran@diw.de}) and we will try to help as well as possible.

\end{document}

%%% Local Variables:
%%% mode: latex
%%% TeX-master: t
%%% End:
