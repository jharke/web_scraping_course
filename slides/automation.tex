\begin{frame}[fragile]{Why browser automation?}
\begin{itemize}
	\item If the content of a page is loaded dynamically (e.g. with JavaScript), using \verb!requests! could yield an ``empty'' source code.
	\item Browser automation is then a way to load the page in an actual browser and let the JavaScript load as if you actually visited the page.
	\item Because this uses an actual browser and a browser driver, this approach is less stable and crashes can occur. Further, loading a page in a browser usually takes more time then loading it in \verb!requests!.
\end{itemize}
\end{frame}

\begin{frame}[fragile]{Example for today}
\begin{itemize}
	\item Let us scrape all future events from the BERA website: \url{https://www.berlin-econ.de/events}.
	\item In order to load all events, we need to click on the bottom buttons to navigate through the results pages.
	\item However, these buttons do not link to a new URL but load content using JavaScript:
\end{itemize}
\begin{verbatim}
<a href="javascript:;" class="item" data-request-success="scroll(0,0)" 
data-request="onEventSearch" 
data-request-update="'@events-list': '#event-results'" 
data-request-data="page:2">Next →</a>
\end{verbatim}
\end{frame}

\begin{frame}[fragile]{Some technical notes}
\begin{itemize}
	\item We will use the \verb!selenium! package
	\begin{itemize}
		\item It allows you to control a browser from a Python script
		\item The documentation can be found here: \url{https://selenium-python.readthedocs.io/}
	\end{itemize}
	\item Besides \verb!selenium!, you need to have an actual browser installed that you are going to use and a compatible browser driver that \verb!selenium! can use to control the browser
	\begin{itemize}
		\item We will use Google's Chrome browser (\url{https://www.google.com/chrome/}) and the corresponding ChromeDriver (\url{http://chromedriver.chromium.org/}). Some parts of the code might have a different syntax for different browsers.
		\item \verb!selenium!'s documentation includes links to drivers for four popular browsers: \url{https://selenium-python.readthedocs.io/installation.html#drivers}
		\item The documentation for the various browser driver types in \verb!selenium! can be found here: \url{https://seleniumhq.github.io/selenium/docs/api/py/api.html}
		\item Make sure that the driver version fits your installed browser version
	\end{itemize}
\end{itemize}
\end{frame}

\begin{frame}[fragile]{First, analyze the HTML code of \url{https://www.berlin-econ.de/events}}
\begin{itemize}
	\item Events are saved in a \verb!<div class='event-results'>! element
	\item Inside this, events for different days are separated by a \verb!<div class='event-date-separator'>! element
	\item The actual events are then saved in a \verb!<div class='ui segments'>! elements, more specifically, in \verb!<div class='ui segment'>! elements
	\item The buttons to navigate to the next results pages are saved in the last element in \verb!<div class='event-results'>! (\verb!<div class='ui pagination menu'>!)
	\item Need a mix of navigating and searching the HTML document
\end{itemize}
\end{frame}

\begin{frame}[fragile]{Approach}
\begin{enumerate}
	\item Load events page in browser
	\item Loop through elements in \verb!<div class='event-results'>!
	\begin{enumerate}
		\item If it is a date, save the date
		\item If it is an event, save the event details
		\item If it is the buttons, click the button for the next page, if available.
	    \item Repeat until no other next page available
	\end{enumerate}
	\item Turn to DataFrame and save
\end{enumerate}
\begin{itemize}
	\item See \verb!automation.ipynb!
\end{itemize}
\end{frame}

\begin{frame}{Notes on interacting with the webpage}
\begin{itemize}
	\item In order to be able to click the button, we need to scroll it into view first
	\item For this, we need to tell \verb!selenium! where the wanted element is and have it scroll there
	\item This can be done e.g. using XPATH syntax
	\item Typical steps are therefore:
	\begin{enumerate}
		\item Find the element in the source code (e.g. \verb!driver.find_element_by_xpath()!, other alternatives here: \url{https://selenium-python.readthedocs.io/locating-elements.html})
		\item Scroll it into view 
		\item Interact with it (scrolling and interacting can both be done using \verb!ActionChains!, see \url{https://seleniumhq.github.io/selenium/docs/api/py/webdriver/selenium.webdriver.common.action_chains.html})
	\end{enumerate}
\end{itemize}

\end{frame}