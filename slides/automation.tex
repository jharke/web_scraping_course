\begin{frame}[fragile]{Why browser automation?}
\begin{itemize}
	\item If the content of a page is loaded dynamically (e.g. with JavaScript), using \verb!requests! could yield an ``empty'' source code.
	\item Browser automation is then a way to load the page in an actual browser and let the JavaScript load as if you actually visited the page.
	\item Because this uses an actual browser and a browser driver, this approach is less stable and crashes can occur. Further, loading a page in a browser usually takes more time then loading it in \verb!requests!.
\end{itemize}
\end{frame}

\begin{frame}[fragile]{Example for today}
\begin{itemize}
	\item Let us scrape all future events from the BERA website: \url{https://www.berlin-econ.de/events}.
	\item In order to load all events, we need to click on the bottom buttons to navigate through the results pages.
	\item However, these buttons do not link to a new URL but load content using JavaScript:
\end{itemize}
\begin{verbatim}
<a href="javascript:;" class="item" data-request-success="scroll(0,0)" 
data-request="onEventSearch" 
data-request-update="'@events-list': '#event-results'" 
data-request-data="page:2">Next →</a>
\end{verbatim}
\end{frame}

\begin{frame}{Approach}
\begin{enumerate}
	\item Load events page in browser
	\item Save information available on all events
	\item Click to the next page
	\item Repeat until no other next page available
\end{enumerate}
\end{frame}