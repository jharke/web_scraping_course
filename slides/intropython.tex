
\begin{frame}
    \frametitle{Your experiences}
    \begin{itemize}
        \item Which tools or programming languages do you use when
        working with data?
        \item Have you used Python before?
    \end{itemize}
\end{frame}

\begin{frame}
    \frametitle{Why Python for Webscraping?}
    \begin{itemize}
        \item<1:> Common web data structures are similar to data
        structures in Python.
        \item<2:> Many Python packages for webscraping and APIs can be
        found.
    \end{itemize}
\end{frame}

\begin{frame}[fragile]
    \frametitle{Python interpreter --- interactive mode}
    \begin{itemize}
        \item input prompt \mintinline{python}{>>>}
        \item comments \mintinline{python}{#}
        \item operators \mintinline{python}{+}, \mintinline{python}{-}, \mintinline{python}{*} and \mintinline{python}{/}
    \end{itemize}
\begin{minted}{python}
>>> 2 + 2
4
>>> 8 / 5  # division always returns a floating point number
1.6
>>> 5 ** 2  # 5 squared
25
\end{minted}
\end{frame}

\begin{frame}[fragile]
    \frametitle{Python data structures}
    \begin{itemize}
        \item Lists (\mintinline{python}{value})\\
\begin{minted}{python}
squares = [1, 4, 9, 16, 25]
\end{minted}
        \item Dictionaries (\mintinline{python}{key: value})\\
\begin{minted}{python}
followers = {'kevin': 15, 'julian': 9}
\end{minted}
    \end{itemize}
\end{frame}

\begin{frame}[fragile]
    \frametitle{Python data structures --- Lists}
\begin{minted}{python}
>>> squares = [1, 4, 9, 16, 25]
>>> squares[0]  # indexing returns the item
1
>>> squares[-1]
25
>>> squares[-3:]  # slicing returns a new list
[9, 16, 25]
>>> squares.append((len(squares) + 1) ** 2)  # using append() method
>>> squares
[1, 4, 9, 16, 25, 36]
\end{minted}
\end{frame}

\begin{frame}[fragile]
    \frametitle{Python data structures --- Dictionaries}
    \begin{itemize}
        \item Dictionaries (\mintinline{python}{key: value})\\
        \item Unlike lists, dictionaries are indexed by keys not by positions.
\begin{minted}{python}
>>> followers = {'kevin': 15, 'julian': 9}
>>> followers['kevin']
15
>>> followers['kevin'] = 16
\end{minted}
    \end{itemize}
\end{frame}

\begin{frame}[fragile]
    \frametitle{Common web data structures}
        For example, JSON nearly identical to combination of Python's dictionaries and lists.
\begin{minted}{json}
[
  {
    "id": "1290837412912998347",
    "followers": "15",
    "name": "Kevin"
  },
  {
    "id": "12908374129734908043",
    "followers": "9",
    "name": "Julian"
  }
]
\end{minted}
\end{frame}

\begin{frame}[fragile]
    \frametitle{Functions}
    \begin{itemize}
        \item  \mintinline{python}{def} defines a function.
        \item Followed by function name with parenthesized sequence of parameters.
        \item Body of function must be indented.
    \end{itemize}
\begin{minted}{python}
>>> def list_append(a, mylist=[]):
...     """Example documentation string:
...     Append value to list. mylist defaults to empty list."""
...     mylist.append(a)
...     return mylist

>>> list_append(97, [99, 98])
[99, 98, 97]

>>> list_append(1)
[1]
\end{minted}
\end{frame}

\begin{frame}[fragile]
    \frametitle{Modules}
    \begin{itemize}
        \item Definitions (functions and variables) can be saved
        in \textbf{modules}. Our example function can be saved under \mintinline{python}{list_operations.py}.
        \item Such modules can be imported into the interpreter,
        scripts, or other modules.
\begin{minted}{python}
import list_operations
list_operations.list_append(1)

from list_operations import list_append
list_append(1)

import list_operations as lo
lo.list_append(1)
\end{minted}
        \item Bad practice: \mintinline{python}{from sound.effects import *}
    \end{itemize}
\end{frame}

\begin{frame}[fragile,fragile]
    \frametitle{Packages}
    \begin{itemize}
        \item Packages structure modules namespace by using dotted
        module names. \mintinline{python}{A.B} designates a submodule
        named \mintinline{python}{B} in a package named
        \mintinline{python}{A}.
\begin{minted}{python}
import matplotlib.pyplot as plt
\end{minted}
        \item Packages can be installed with pip.
\begin{minted}{bash}
pip install matplotlib
\end{minted}
        \item A common convention is to have a list of packages in requirements.txt:
\begin{minted}{bash}
pip install -r requirements.txt
\end{minted}
    \end{itemize}
\end{frame}

\begin{frame}[fragile]
    \frametitle{Files}
    \begin{itemize}
        \item Reading and writing files
\begin{minted}{python}
f = open('workfile', 'r') # read-only
f = open('workfile', 'w') # write-only
f = open('workfile', 'a') # appending

f = open('workfile') # read-only, as mode defaults to 'r'
\end{minted}
        \item Closing files
\begin{minted}{python}
f.close() # Manually close a file

with open('workfile') as f: # with closes file "automatically"
     read_data = f.read()
\end{minted}
        \item Several packages offer file operations
        \mintinline{python}{pandas.read_csv()}
    \end{itemize}
\end{frame}

% \begin{frame}
%     \frametitle{Further reading}
%     \begin{itemize}
%         \item
%     \end{itemize}
% \end{frame}
