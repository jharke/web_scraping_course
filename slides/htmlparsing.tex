\begin{frame}{HTML parsing}
\begin{itemize}
	\item After obtaining the HTML source code, how to obtain the information required?
	\item If the HTML code is well-structured and its tags have (more or less) unique names, we can navigate the HTML elements to get the information we want.
	\item The \mintinline{python}{beautifulsoup4} package converts the HTML code into a Python object that can be navigated using properties and functions.
\end{itemize}
\end{frame}

\begin{frame}[fragile]{Some HTML terms}
\begin{itemize}
	\item Consider \mintinline{html}{<a href="http://www.bccp-berlin.de" target="_blank">BCCP</a>}	 
	\item HTML Elements
	\begin{itemize}
		\item The entire thing is an HTML element. Specifically, it is a link leading to the BCCP website and displayed as "BCCP".
		\item HTML elements usually consist of a start tag and an end tag.
	\end{itemize}
	\item HTML Tags
	\begin{itemize}
		\item The start tag of the element above is  \mintinline{html}{<a>} and the end tag is the corresponding  \mintinline{html}{</a>}
		\item Start tag can and sometimes must contain attributes.
	\end{itemize}
	\item HTML Attributes
	\begin{itemize}
		\item The  \mintinline{html}{<a>} tag contains the attribute  \mintinline{html}{href} and  \mintinline{html}{target}.  \mintinline{html}{href} specifies the destination to which the link should lead and  \mintinline{html}{target="\_blank"} specifies that the link should be opened in a new window.
		\item For web scraping purposes, the attributes  \mintinline{html}{class} and  \mintinline{html}{id} are usually useful as these are often used to identify certain (groups of) elements.
	\end{itemize}
\end{itemize}
\end{frame}

\begin{frame}{Basic HTML documents structure}
\begin{itemize}
	\item HTML documents have a tree-like/nested structure
	\item Elements can contain various levels of sub-elements that in the end contain some content
\end{itemize}
\end{frame}

\begin{frame}[fragile]{HTML document example}
\begin{minted}{html}
<html>
 <body>
  <p>
   Hi all!
  </p>
  <p>
   Do you know <a href="http://www.bccp-berlin.de" target="_blank">BCCP</a>?
  </p>
 </body>
</html>
\end{minted}
\end{frame}

\begin{frame}[fragile]{Tree structure}
\begin{center}
\begin{tikzpicture}
\draw (0,0) node[above] {\mintinline{html}{<html>}};
\draw (0,0) -- (0,-1) node[below] {\mintinline{html}{<body>}};
\draw (0,-1.5) -- (0,-2);
\draw (-2,-2.5) node[below] {\mintinline{html}{<p>}} -- (-2,-2) -- (2,-2) -- (2,-2.5) node[below] {\mintinline{html}{<p>}};	
\draw (-2,-3) -- (-2,-4) node[below] {Hi all!};
\draw (2,-3) -- (2,-3.5);
\draw (1,-4) node[below] {Do you know} -- (1,-3.5) -- (3,-3.5) -- (3,-4) node[below] {\mintinline{html}{<a>}} -- (3,-3.5) -- (5,-3.5) -- (5,-4) node[below] {?};
\draw (3,-4.5) -- (3,-5) node[below] {BCCP};
\end{tikzpicture}
\end{center}
\end{frame}

\begin{frame}[fragile, allowframebreaks]{A note on XPath}
\begin{itemize}
	\item XPath is a syntax to address specific elements in an HTML document
	\item Most important expressions:
	\begin{itemize}
		\item \mintinline{html}{//div}: Select all \mintinline{html}{<div>} tags in the HTML document
		\item \mintinline{html}{//div//span}: Select all \mintinline{html}{<span>} tags in the HTML document that are descendants of a \mintinline{html}{<div>} tag, e.g. \mintinline{html}{<div><ul><li><span></span></li></ul></div>}
		\item \mintinline{html}{//div/span}: Select all \mintinline{html}{<span>} tags in the HTML document that are children of a \mintinline{html}{<div>} tag, e.g. \mintinline{html}{<div><span></span></div>} but not \mintinline{html}{<div><ul><li><span></span></li></ul></div>}

\framebreak
		
		\item \mintinline{html}{//div/span[@class='bccp']}: Select all \mintinline{html}{<span>} tags in the HTML document that are children of a \mintinline{html}{<div>} tag and whose \mintinline{html}{class} attribute takes the value `bccp', e.g. \mintinline{html}{<div><span class='bccp'></span></div>} but not \mintinline{html}{<div><span class='bccp course'></span></div>}
		\item \mintinline{html}{//div/span[contains(@class,'bccp')]}: Select all \mintinline{html}{<span>} tags in the HTML document that are children of a \mintinline{html}{<div>} tag and whose \mintinline{html}{class} attribute value contains `bccp', e.g. \mintinline{html}{<div><span class='bccp course'></span></div>}
		\item When inspecting source codes in Chrome, you can use XPath to find elements
		\item Find more here: \url{https://www.w3schools.com/xml/xpath_intro.asp}
	\end{itemize}
\end{itemize}

\end{frame}

\begin{frame}[fragile]{General steps}
\begin{enumerate}
	\item Load a web page and get the source code: Use \mintinline{python}{requests} for static and \mintinline{python}{selenium} for dynamic websites
	\item Convert (``parse'') the source code into a soup object: Use \mintinline{python}{beautifulsoup4}
	\item Navigate/search the soup object to get the information you want
\end{enumerate}
\end{frame}

\begin{frame}{Example for today}
\begin{itemize}
	\item Let's scrape the details of all upcoming BCCP events: \url{http://www.bccp-berlin.de/events/all-events/}
	\item Steps:
		\begin{enumerate}
			\item Analyze HTML structure
			\item Load source code
			\item Loop through events available on the front page and save details
			\item Create DataFrame from this
			\item Loop through and load individual event pages to save further details
		\end{enumerate}
\end{itemize}
\end{frame}

\begin{frame}[fragile,allowframebreaks]{Analyzing the HTML structure}
\begin{itemize}
	\item Open \url{http://www.bccp-berlin.de/events/all-events/} in a browser and inspect the source code
	\item Information on events saved in div elements
\end{itemize}
\begin{minted}{html}
<div class="eventList">
...
<div class="event-list-item event-type1">...</div>
...
<div class="event-list-item event-type2">...</div>
...
</div>
\end{minted}

\framebreak

\begin{itemize}
	\item Details are saved in sub-elements in each \mintinline{html}{//div[contains(@class,'event-list-item')]} element
\end{itemize}
\begin{minted}{html}
<div class="event-list-item event-type1">
 <div class="top-bar">
  <span class="date single">June 27, 2019</span>
  <span class="b-events__item__type">Seminar</span>
 </div>
 <div class="b-events__item__inner">
  <div class="content">
   <div class="genres">Berlin Behavioral Economics Seminar</div>
   <h2 class="eventHeader">
    <a href="/events/all-events/events-detail/
    felix-holzmeister-university-of-innsbruck/">
     Felix Holzmeister (University of Innsbruck)
    </a>
   </h2>
   <div class="teaser">Delegated Decision Making in Finance</div>
   <div class="location">
    <strong class="label">Location</strong>
    <div class="address">
     <span class="name">WZB</span>
     <span class="address">Reichpietschufer 50, Room B001</span>
     <span class="zip">10785</span>
     <span class="place">Berlin</span>
    </div>
   </div>
   <div class="time">
    <strong class="label">Time</strong>
    <span>16:45–18:00</span>
   </div>
  </div>
  <div class="button detail">
   <a title="Felix Holzmeister (University of Innsbruck)" 
   href="/events/all-events/events-detail/
   felix-holzmeister-university-of-innsbruck/">
    Event Details
   </a>
  </div>
 </div>
</div>
\end{minted}
\end{frame}

\begin{frame}[fragile]{From website to Python soup}
See \mintinline{bash}{htmlparsing.ipynb}.
\begin{enumerate}
	\item \mintinline{python}{requests}: Load website and save source code as string
\begin{minted}{python}
url = "http://www.bccp-berlin.de/events/all-events" # URL to BCCP events page
r = requests.get(url) # Load URL
srccode = r.text # Save source code
\end{minted}
	\item \mintinline{python}{beautifulsoup4}: Take source string and parse to get soup object
		\begin{itemize}
			\item There are three different parsers: \mintinline{python}{html.parser}, \mintinline{python}{lxml}, \mintinline{python}{html5lib}
			\item Differences are discussed here: \url{https://www.crummy.com/software/BeautifulSoup/bs4/doc/#installing-a-parser}
			\item I usually use \mintinline{python}{lxml}
		\end{itemize}
\begin{minted}{python}
soup = BeautifulSoup(srccode, "lxml")
\end{minted}
\end{enumerate}
\end{frame}

\begin{frame}[fragile]{Some BeautifulSoup functions}
\begin{itemize}
	\item Look at the very good documentation: \url{https://www.crummy.com/software/BeautifulSoup/bs4/doc/}
	\item You can either \textit{search} the document:
	\begin{itemize}
		\item \mintinline{python}{.find_all()}: Find all elements that match a certain condition. Returns a list.
		\item \mintinline{python}{.find()}: Same as \mintinline{python}{.find_all()} but only returns first match.
	\end{itemize}
	\item If unique tag names are not available, \textit{navigation} of the HTML tree rather than searching it is possible, e.g.:
	\begin{itemize}
		\item Vertically: \mintinline{python}{.parent}, \mintinline{python}{.parents}, \mintinline{python}{.children}
		\item Horizontally: \mintinline{python}{.next_sibling}, \mintinline{python}{.previous_sibling}
	\end{itemize}
\end{itemize}
\end{frame}