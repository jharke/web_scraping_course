\begin{frame}{Introduction to Webscraping}
\begin{itemize}
	\item Basic idea: Turn information on website to structured data
	% Screenshot of webpage and corresponding data
	\item Typical workflow:
	\begin{enumerate}
		\item Look at website to decide best approach
		\begin{itemize}
			\item Is an Application Programming Interface (API) available?
			\item Do the HTML elements have fixed names?
			\item Does the page load statically or dynamically?
		\end{itemize}
		\item Download information from URL
		\item Turn information into structured data and save
	\end{enumerate}
\end{itemize}
\end{frame}

\begin{frame}{Some concepts}
\begin{itemize}
	\item APIs
	\item HTML parsing vs text matching
	\item Static vs dynamic websites
\end{itemize}
\end{frame}

\begin{frame}{APIs}
% Example of website screenshot and JSON result
\begin{itemize}
	\item If available, a convenient way to get pre-structured data (usually JSON or XML).
	\item Example: See $0\_intro\_webscraping.ipynb$
\end{itemize}
\end{frame}

\begin{frame}{HTML parsing}
% Example of website screenshot and HTML code
\begin{itemize}
	\item Use structure of HTML code to find needed information.
	\item Works best if the code is well-structured and element names are fixed.
	\item Example: See $0\_intro\_webscraping.ipynb$
\end{itemize}
\end{frame}

\begin{frame}{Text pattern matching}
% Example of website screenshot and HTML code with random names
\begin{itemize}
	\item If the HTML code is not well-structured or names change, text pattern matching is an alternative.
	\item Idea: Take text from (parts of) a page and find needed information by matching a regular expression
	\item Example: See $0\_intro\_webscraping.ipynb$
\end{itemize}
\end{frame}

\begin{frame}{Static vs dynamic websites}
% Example of website screenshot and HTML code with random names
\begin{itemize}
	\item On static websites, the entire content is loaded immediately.
	\item On dynamic websites, content may load subsequently or after user action, making them usually more complicated to scrape.
	% Show examples of static and dynamic websites and what happens if loading with requests
	\item Example: See $0\_intro\_webscraping.ipynb$
\end{itemize}
\end{frame}

\begin{frame}{Important Python packages}
\begin{itemize}
	\item {\tt requests}: To load URL and recover source code (for static web pages)
	\item {\tt beautifulsoup4}: To turn HTML code to navigable Python object
	\item {\tt selenium}: To automate browsers
	\item {\tt pandas}: To create DataFrames
\end{itemize}

\end{frame}